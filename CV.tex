%!TEX TS-program = xelatex
%!TEX encoding = UTF-8 Unicode
\documentclass[letterpaper, 10 pt]{mycv}
\title{Tyler Morrison}
\date{\today}
\name{Tyler N.}{Morrison}
\position{Graduate Research Assistant}
\mobile{913-944-2055}
\email{morrison.730@osu.edu}
\homepage{tyler-morrison.com}
%\github{tymo77}
%\gitlab{}
%\stackoverflow{}
\linkedin{tymo77}
%\twitter{}
%\skype{}
%\reddit{}
%\xing{}
%\extrainfo{}
\quote{``Let's see if this works.''}
%\recipient{name}{address}
\photo[rectangle,right,noedge]{images/head_nobg}

% Set false if you don't want to highlight section with awesome color
\setbool{acvSectionColorHighlight}{false}

% Specify the location of any included fonts
\fontdir[fonts/]

% Colors for text
% Uncomment if you would like to specify your own color
% \definecolor{darktext}{HTML}{414141}
% \definecolor{text}{HTML}{333333}
% \definecolor{graytext}{HTML}{5D5D5D}
% \definecolor{lighttext}{HTML}{999999}}

\begin{document}
	% Print the header with above personal informations
	% Give optional argument to change alignment(C: center, L: left, R: right)
	\makecvheader[L]
	
	% Print the footer with 3 arguments(<left>, <center>, <right>)
	% Leave any of these blank if they are not needed
	\makecvfooter{\today}{Tyler N. Morrison~~~|+|~~~Curriculum Vitae}{\thepage}
%	
%	%-------------------------------------------------------------------------------
%	%	CV/RESUME CONTENT
%	%	Each section is imported separately, open each file in turn to modify content
%	%-------------------------------------------------------------------------------
	%-------------------------------------------------------------------------------
%	SECTION TITLE
%-------------------------------------------------------------------------------
\cvsection{Education}


%-------------------------------------------------------------------------------
%	CONTENT
%-------------------------------------------------------------------------------
\begin{cventries}
	
	\cveducation
	{Ph.D. Candidate in Mechanical Engineering} % Degree
	{The Ohio State University} % Institution
	{Columbus, OH, USA} % Location
	{Aug. 2017 --- est. Dec. 21 or May 22} % Date(s)
	{
		\begin{cvitems} % Description(s) bullet points
			\item {GPA: 4.00/4.00}
			\item {Candidacy Exam Passed in Nov. 2020}
			\item {Distinguished University Fellow}
			\item {Department of Mechanical Engineering Supplementary Award}
		\end{cvitems}
	}
	{images/blocko.png}

%---------------------------------------------------------
	\cveducation
    {B.S. Mechanical Engineering} % Degree
    {The University of Tulsa} % Institution
    {Tulsa, OK, USA} % Location
    {Aug. 2013 --- May 2017} % Date(s)
    {
      \begin{cvitems} % Description(s) bullet points
      	\item {GPA: 4.00/4.00}
        \item {Chapman Presidential Scholar}
        \item {Vision Scholar}
      \end{cvitems}
    }
	{images/tulogo2.png}
%---------------------------------------------------------
\end{cventries}

	%-------------------------------------------------------------------------------
%	SECTION TITLE
%-------------------------------------------------------------------------------
\cvsection{Research Experience}


%-------------------------------------------------------------------------------
%	CONTENT
%-------------------------------------------------------------------------------
\begin{cventries}

%---------------------------------------------------------
  \cventry
    {Graduate Research Assistant} % Job title
    {Design, Innovation and Simulation Lab (DISL)} % Organization
    {The Ohio State University, Columbus, OH, USA} % Location
    {August 2017 --- Present} % Date(s)
    {
      \begin{cvitems} % Description(s) of tasks/responsibilities
        \item{Developed software, planned, and conducted experiments for interdisciplinary research on attention to preview in human drivers.}
        \item{Modeled, tested, and analyzed variable stiffness links (VSLs) for use in corobots.}
        \item{Mentored high school, undergraduate, and MS students on research projects.}
        \item{Administered upkeep of lab server and rapid-prototyping equipment.}
      \end{cvitems}
    }

%---------------------------------------------------------
  \cventry
    {REU Research Assistant} % Job title
    {NSF Interfaces and Surfaces REU} % Organization
    {Clemson University, Clemson, SC, USA} % Location
    {May 2016 --- August 2016} % Date(s)
    {
      \begin{cvitems} % Description(s) of tasks/responsibilities
        \item {Conducted numerical simulations of hydrogel membranses under illumination.}
        \item {Modeled and implemented code for numerical simulation of magnetically heated gels with cooling effects.}
        \item {Mentored incoming MS student on simulation software and high-performance-computing.}
      \end{cvitems}
    }

%---------------------------------------------------------
  \cventry
    {Undergraduate Research Assistant} % Job title
    {Biological Robotics at Tulsa Lab (BRAT Lab)} % Organization
    {The University of Tulsa, Tulsa, OK, USA} % Location
    {May 2015 --- July 2017} % Date(s)
    {
      \begin{cvitems} % Description(s) of tasks/responsibilities
        \item {Pursued independent research on grasping and manipulation with quadruped robots.}
        \item {Developed method optimal foot-shuffle algorithm for quadruped stabilization under disturbances and body-position constraints.}
        \item {Developed interactive 3D model of quadruped kinematics and tip-over stability.}
      \end{cvitems}
    }

\end{cventries}
	%-------------------------------------------------------------------------------
%	SECTION TITLE
%-------------------------------------------------------------------------------
\cvsection{Peer-Reviewed Journal Articles}


%-------------------------------------------------------------------------------
%	CONTENT
%-------------------------------------------------------------------------------
\begin{cventries}

%---------------------------------------------------------

\cvarticle{\me, Hai-Jun Su}{Stiffness Modeling of a Variable Stiffness Compliant Link}{Mechanism and Machine Theory}{2020}{Accepted}{}

\cvarticle{\me, Emanuele Rizzi, Omer Turkkan, Richard Jagacinski, Hai-Jun Su, Junmin Wang}{Drivers' Spatio-Temporal Attentional Distributions Are Influenced by Vehicle Dynamics and Displayed Point of View}{Human Factors}{2020}{Published}{10.1177/0018720820902879}

\cvarticle{Richard Jagacinski, Emanuele Rizzi, Benjamin Bloom, Omer Turkkan, \me, Hai-Jun Su, Junmin Wang}{Drivers’ Attentional Instability on a Winding Roadway}{IEEE Transactions on Human-Machine Systems}{2019}{Published}{10.1109/THMS.2019.2906612}

\cvarticle{Oksana Savchak, Konstantin Kornev, \me, Olga Kuksenok}{Controlling Deformations of Gel-based Composites by Electromagnetic Signals within GHz Frequency Range}{Soft Matter}{2018}{Published}{10.1039/C8SM01207E}

%---------------------------------------------------------
\end{cventries}

	\newpage
	%-------------------------------------------------------------------------------
%	SECTION TITLE
%-------------------------------------------------------------------------------
\cvsection{Conference Papers}


%-------------------------------------------------------------------------------
%	CONTENT
%-------------------------------------------------------------------------------
\begin{cventries}

%---------------------------------------------------------

\cvpresentation{\me, Chunhui Li, Xu Pei, Hai-Jun Su}{A Novel Rotating Beam Link for Variable Stiffness Robotics Arms}{IEEE International Conference on Robotics and Automation}{Montreal,~CA}{May~2019}{In~Press}{}

%---------------------------------------------------------
\end{cventries}

	%-------------------------------------------------------------------------------
%	SECTION TITLE
%-------------------------------------------------------------------------------
\cvsection{Other Selected Research Presentations}


%-------------------------------------------------------------------------------
%	CONTENT
%-------------------------------------------------------------------------------
\begin{cventries}

%---------------------------------------------------------

\cvpresentation{\me, Joshua Schultz}{Optimal shuffles to prevent quadruped tipover during cooperation}{Robotics: Science and Systems Workshop on Design and Control of Small Legged Robots}{Pittsburgh,~PA,~USA}{June~2018}{Poster~+~Oral}{}
 
%\cvpresentation{Oksana Sachak, Yao Xiong, \me, Konstantin Kornev, Olga Kuksenok}{Magnonics in hydrogels: modeling and magnetomechanical effects in GHz frequency range}{MRS Spring Meeting: Computer-Based Modeling and Experiment for the Design of Soft Materials Symposium}{Phoenix,~AZ,~USA}{April~2017}{Oral}{}

%\cvpresentation{\me, Joshua Schultz}{Algorithms for Shuffling Foot Placements to Maintain Stability of a Quadruped Robot Engaged in a Cooperative Task}{The University of Tulsa Student Research Colloquium Mechanical: Engineering Session}{Tulsa,~OK,~USA}{March~2017}{Oral}{}

\cvpresentation{\me, Olga Kuksenok}{Numerical Simulations of NIPA Gel Membranes Exposed to Heat and Light}{Clemson Undergraduate Research Symposium}{Clemson,~SC,~USA}{July~2016}{Poster}{}

%\cvpresentation{\me, Joshua Schultz}{Investigation into Coordinated Gaits of Quadruped Robots Engaged in Grasping and Manipulation with Applications in Search and Rescue}{The University of Tulsa Student Research Colloquium: Mechanical Engineering Session}{Tulsa,~OK,~USA}{March~2016}{Oral}{}
%---------------------------------------------------------
\end{cventries}

	%-------------------------------------------------------------------------------
%	SECTION TITLE
%-------------------------------------------------------------------------------
\cvsection{Other Presentations}


%-------------------------------------------------------------------------------
%	CONTENT
%-------------------------------------------------------------------------------
\begin{cventries}

%---------------------------------------------------------
\cvpresentation{Abdulhamid Aljaber, Sultan Al-Nabhani, Nick Criser, Brian Hall, Joel Kapp, \me, Drake Norman, Alex Price, Tommy Weissert}{The Tulsa Children’s Museum Petroleum Exhibit Ball Lift}{The University of Tulsa Senior Projects}{Tulsa,~OK,~USA}{April~2017}{Oral}{}

\cvpresentation{Brian Hall, \me, Alex Price}{The Tulsa Children’s Museum Auger Ball Lift}{The University of Tulsa Department of Mechanical Engineering Advisory Board Meeting}{Tulsa,~OK,~USA}{March~2017}{Oral}{}

\cvpresentation{\me, Joshua Schultz}{Algorithms for Shuffling Foot Placements to Maintain Stability of a Quadruped Robot Engaged in a Cooperative Task}{The University of Tulsa Student Research Colloquium Mechanical: Engineering Session}{Tulsa,~OK,~USA}{March~2017}{Oral}{}

\cvpresentation{\me, Olga Kuksenok}{Numerical Simulations of NIPA Gel Membranes Exposed to Heat and Light}{Clemson Undergraduate Research Symposium}{Clemson,~SC,~USA}{July~2016}{Poster}{}

\cvpresentation{\me, Joshua Schultz}{Investigation into Coordinated Gaits of Quadruped Robots Engaged in Grasping and Manipulation with Applications in Search and Rescue}{The University of Tulsa Student Research Colloquium: Mechanical Engineering Session}{Tulsa,~OK,~USA}{March~2016}{Oral}{}

%---------------------------------------------------------
\end{cventries}

	%-------------------------------------------------------------------------------
%	SECTION TITLE
%-------------------------------------------------------------------------------
\cvsection{Skills and Experience}


%-------------------------------------------------------------------------------
%	CONTENT
%-------------------------------------------------------------------------------
\begin{cvskills}

%---------------------------------------------------------
  \cvskill
    {Programming} % Category
    {Mathematica, MATLAB, Python, JAVA, C, VBA, \texttt{\LaTeX}} % Skills

%---------------------------------------------------------
  \cvskill
    {Modeling and Design Software} % Category
    {Solidworks, Revit, AutoCAD, Adobe Illustrator} % Skills

%---------------------------------------------------------
  \cvskill
    {Simulation} % Category
    {ANSYS, ABAQUS, Solidworks Simulation} % Skills

%---------------------------------------------------------
  \cvskill
    {Hardware} % Category
    {Arduino, Raspberry Pi} % Skills
    
%---------------------------------------------------------
  \cvskill
    {Rapid Prototyping} % Category
    {3D-Printing, Laser Cutting, Plasma Cutting} % Skills

%---------------------------------------------------------
\end{cvskills}

	%-------------------------------------------------------------------------------
%	SECTION TITLE
%-------------------------------------------------------------------------------
\cvsection{Additional Experience}


%-------------------------------------------------------------------------------
%	CONTENT
%-------------------------------------------------------------------------------
\begin{cventries}

%---------------------------------------------------------
  \cventry
    {Student Grader} % Job title
    {The University of Tulsa Mechanical Engineering Department} % Organization
    {Tulsa, OK, USA} % Location
    {Spring 2016 --- Spring 2017} % Date(s)
    {
      \begin{cvitems} % Description(s) of tasks/responsibilities
        \item{Mechanics of Materials (ME 3023)}
        \item{Instrumentation and Measurement (ME 3053)}
      \end{cvitems}
    }

%---------------------------------------------------------
\cventry
	{Mechanical Engineering Intern} % Job title
	{Mechanical Department, Aviation and Federal Division, Burns and McDonnell} % Organization
	{Kansas City, MO, USA} % Location
	{Summer 2015} % Date(s)
	{
		\begin{cvitems} % Description(s) of tasks/responsibilities
			\item{Assisted in designing HVAC and plumbing systems at Tinker Air Force Base, Portland International Airport, The Sampson School At Guantanamo Bay Naval Base}
		\end{cvitems}
	}

%---------------------------------------------------------
\cventry
	{Jobsite Administrative Assistant} % Job title
	{Navitas ESCO} % Organization
	{Olathe, KS, USA} % Location
	{Summer 2014} % Date(s)
	{}

%---------------------------------------------------------
%\cventry
%	{} % Job title
%	{Dairy Queen} % Organization
%	{Overland Park, KS, USA} % Location
%	{2012 --- 2013} % Date(s)
%	{}
\end{cventries}
	%-------------------------------------------------------------------------------
%	SECTION TITLE
%-------------------------------------------------------------------------------
\cvsection{Extracurricular Activity}


%-------------------------------------------------------------------------------
%	CONTENT
%-------------------------------------------------------------------------------
\begin{cventries}
	
%---------------------------------------------------------
	\cventry
	{Involved Member of Oklahoma Beta Chapter} % Affiliation/role
	{Tau Beta Pi} % Organization/group
	{Tulsa, OK, USA} % Location
	{2015 --- 2017} % Date(s)
	{
		\begin{cvitems} % Description(s) of experience/contributions/knowledge
			\item {Helped lead initiation of new members and organize induction ceremonies and fellowship activities.}
		\end{cvitems}
	}

%---------------------------------------------------------
  \cventry
    {University of Tulsa Team Member and Individual Competitor} % Affiliation/role
    {Mathematical Association of America Putnam Competition} % Organization/group
    {Tulsa, OK, USA} % Location
    {2014, 2015} % Date(s)
    {
      \begin{cvitems} % Description(s) of experience/contributions/knowledge
        \item {Personal best score of 11, ranks in top 21\% of mathematicians in North America. (Median score is zero points)}
        \item {Trained with Dr. Christian Constanda.}
        \item {Resigned from team to focus on research.}
      \end{cvitems}
    }

%---------------------------------------------------------
\cventry
	{Eagle Scout} % Affiliation/role
	{Boy Scouts of America} % Organization/group
	{Kansas City, KS, USA} % Location
	{2001 --- 2013} % Date(s)
	{
		\begin{cvitems} % Description(s) of experience/contributions/knowledge
			\item {Eagle Scout service project: leading and organizing a project to build bookshelves for a library system at a home for troubled children.}
			\item {Member of U.S. BSA Delegation to the 2011 World Scout Jamboree in Sweden.}
			\item {Heart of America Council, Trailhead District, Troop 92 \& Pack 3449.}
		\end{cvitems}
	}
%---------------------------------------------------------
\end{cventries}

	%-------------------------------------------------------------------------------
%	SECTION TITLE
%-------------------------------------------------------------------------------
\cvsection{Selected Honors \& Awards}


%-------------------------------------------------------------------------------
%	SUBSECTION TITLE
%-------------------------------------------------------------------------------
\cvsubsection{Graduate}


%-------------------------------------------------------------------------------
%	CONTENT
%-------------------------------------------------------------------------------
\begin{cvhonors}

%---------------------------------------------------------
  \cvhonor
    {Distinguished University Fellow} % Award
    {The Ohio State University} % Event
    {Columbus, OH, USA} % Location
    {2017} % Date(s)

%---------------------------------------------------------
  \cvhonor
    {Department Supplementary Fellowship Award} % Award
    {The Ohio State University} % Event
    {Columbus, OH, USA} % Location
    {2017} % Date(s)

%---------------------------------------------------------
\end{cvhonors}


%-------------------------------------------------------------------------------
%	SUBSECTION TITLE
%-------------------------------------------------------------------------------
\cvsubsection{Undergraduate}


%-------------------------------------------------------------------------------
%	CONTENT
%-------------------------------------------------------------------------------
\begin{cvhonors}

%---------------------------------------------------------
  \cvhonor
    {College of Engineering and Natural Sciences Steven J. Bellovich Medal} % Award
    {The University of Tulsa} % Event
    {Tulsa, OK, USA} % Location
    {2017} % Date(s)

%---------------------------------------------------------
  \cvhonor
    {Sidney Born Award in Mechanical Engineering} % Award
    {The University of Tulsa} % Event
    {Tulsa, OK, USA} % Location
    {2017} % Date(s)

%---------------------------------------------------------
  \cvhonor
    {Senior Project -- Most Valuable Team Member} % Award
    {The University of Tulsa, Mechanical Engineering Dept.} % Event
    {Tulsa, OK, USA} % Location
    {2017} % Date(s)

%---------------------------------------------------------
%  \cvhonor
%    {Best Senior Project Team: Runner-Up} % Award
%    {The University of Tulsa Mechanical Engineering Senior Projects} % Event
%    {Tulsa, OK, USA} % Location
%    {2017} % Date(s)

%---------------------------------------------------------
  \cvhonor
    {Nominee for National Barry Goldwater Scholarship} % Award
    {The University of Tulsa} % Event
    {Tulsa, OK, USA} % Location
    {2017} % Date(s)
    
%%---------------------------------------------------------
%\cvhonor
%	{President's Honor Roll} % Award
%	{The University of Tulsa} % Event
%	{Tulsa, OK, USA} % Location
%	{2013 --- 2017} % Date(s)
%% 
%%---------------------------------------------------------
%\cvhonor
%	{National Merit Finalist} % Award
%	{National Merit Scholarship Corporation} % Event
%	{} % Location
%	{2013} % Date(s)

%---------------------------------------------------------
\end{cvhonors}

	%-------------------------------------------------------------------------------
%	SECTION TITLE
%-------------------------------------------------------------------------------
\cvsection{Attended Conferences}


%-------------------------------------------------------------------------------
%	CONTENT
%-------------------------------------------------------------------------------
\begin{cvitems}

	\vspace{4.0mm}
%---------------------------------------------------------
  	\item
    {Robotics: Science and Systems; June 26-30, 2018; Pittsburgh, PA, USA}

%---------------------------------------------------------
\end{cvitems}
\vspace{4.0mm}

	%-------------------------------------------------------------------------------
%	SECTION TITLE
%-------------------------------------------------------------------------------
\cvsection{Peer-Review Assistance}


%-------------------------------------------------------------------------------
%	CONTENT
%-------------------------------------------------------------------------------
\begin{cvitems}

	\vspace{4.0mm}
%---------------------------------------------------------
  	\item{IROS 2018, Mechanism and Machine Theory, RoboSoft 2020}
%---------------------------------------------------------
\end{cvitems}

\vspace{4.0mm}

%	%-------------------------------------------------------------------------------
%	SECTION TITLE
%-------------------------------------------------------------------------------
\cvsection{Passions}


%-------------------------------------------------------------------------------
%	CONTENT
%-------------------------------------------------------------------------------
\begin{cvitems}

	\vspace{4.0mm}
%---------------------------------------------------------
  	\item
    {Running and Biking, Dog Fostering and Training, Medical Spouse}
%---------------------------------------------------------
	%\item
	%{Dog Fostering and Training}
%---------------------------------------------------------
	%\item
	%{Medical Spouse.}


%---------------------------------------------------------
\end{cvitems}

	
	
%	%-------------------------------------------------------------------------------
%	SECTION TITLE
%-------------------------------------------------------------------------------
\cvsection{Writing}


%-------------------------------------------------------------------------------
%	CONTENT
%-------------------------------------------------------------------------------
\begin{cventries}

%---------------------------------------------------------
  \cventry
    {Founder \& Writer} % Role
    {A Guide for Developers in Start-up} % Title
    {Facebook Page} % Location
    {Jan. 2015 - PRESENT} % Date(s)
    {
      \begin{cvitems} % Description(s)
        \item {Drafted daily news for developers in Korea about IT technologies, issues about start-up.}
      \end{cvitems}
    }

%---------------------------------------------------------
  \cventry
    {Undergraduate Student Reporter} % Role
    {AhnLab} % Title
    {S.Korea} % Location
    {Oct. 2012 - Jul. 2013} % Date(s)
    {
      \begin{cvitems} % Description(s)
        \item {Drafted reports about IT trends and Security issues on AhnLab Company magazine.}
      \end{cvitems}
    }

%---------------------------------------------------------
\end{cventries}

%	%-------------------------------------------------------------------------------
%	SECTION TITLE
%-------------------------------------------------------------------------------
\cvsection{Program Committees}


%-------------------------------------------------------------------------------
%	CONTENT
%-------------------------------------------------------------------------------
\begin{cvhonors}

%---------------------------------------------------------
  \cvhonor
    {Problem Writer} % Position
    {2016 CODEGATE Hacking Competition World Final} % Committee
    {S.Korea} % Location
    {2016} % Date(s)

%---------------------------------------------------------
  \cvhonor
    {Organizer \& Co-director} % Position
    {1st POSTECH Hackathon} % Committee
    {S.Korea} % Location
    {2013} % Date(s)

%---------------------------------------------------------
  \cvhonor
    {Staff} % Position
    {7th Hacking Camp} % Committee
    {S.Korea} % Location
    {2012} % Date(s)

%---------------------------------------------------------
  \cvhonor
    {Problem Writer} % Position
    {1st Hoseo University Teenager Hacking Competition} % Committee
    {S.Korea} % Location
    {2012} % Date(s)

%---------------------------------------------------------
  \cvhonor
    {Staff \& Problem Writer} % Position
    {JFF(Just for Fun) Hacking Competition} % Committee
    {S.Korea} % Location
    {2012} % Date(s)

%---------------------------------------------------------
\end{cvhonors}

\end{document}